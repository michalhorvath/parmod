\documentclass[a4paper]{article}
\usepackage[a4paper, tmargin=0.7in, bmargin=1in, lmargin=1in, rmargin=1in]{geometry}
\usepackage[utf8]{inputenc}
\usepackage[slovak]{babel}
\usepackage{indentfirst}

\usepackage{graphicx}
\graphicspath{{./}}

\title{Špecifikácia Parmod}
\author{Michal Horváth }
\date{9.\ marca\ 2023}

\begin{document}

\maketitle

%===========================================================================
%===========================================================================
%===========================================================================
\section{Základný popis internetovej aplikácie}
\textbf{Parmod} je webová aplikácia určená pre zdielanie parametrických 3D modelov.
Prepája ľudí, ktorí dokážu vytvárať parametrické dizajny 
v open-source softvéri OpenSCAD,
s ľudmi, ktorí tieto modely využívajú,
pričom im umožnuje modely modifikovať
cez webové rozhranie zmenou hodnôt predurčených parametrov.

\

Kľúčové pojmy:
\begin{itemize}

\item
\textbf{Dizajn} určuje abstraktnú triedu modelov. 
Je definovaný OpenSCAD skriptom a špecifikovaním parametrov.
Dizajnu je možné priradiť obrázok a popis.
Jedná sa o hlavný prvok aplikácie.

\item
\textbf{Parameter} je určený názvom, názvom premennej, typom a predvolenou hodnotou.
Názov je zobrazený v aplikácii a popisuje daný parameter.
Názvom premennej určuje premennú OpenSCAD skriptu, ktorú parameter modifikuje.
Typ obmedzuje aké hodnoty môžu byť parametru priradené.
Typ môže byť celé číslo, desatinné číslo, 
celé číslo ohraničené intervalom, desatinné číslo ohraničené intervalom, 
ľubovolný textový reťazec, textový reťazec zo zoznamu.
Predvolená hodnota určuje hodnotu, ktorú parameter nadobudne
ak parametru nebude priradená konkrétna hodnota.

\item
\textbf{Model} je vygenerovaný z konkrétne zvoleného dizajnu
a priradením konkrétnych hodnôt parametrov, ktoré dizajn špecifikuje.
Po vygenerovaní sa dá v aplikácii prezerať a stiahnuť vo formáte \texttt{.stl}.
Model zostáva dostupný v aplikácii, pre zrýchlenie opakovaného generovania.

\item
\textbf{Katalóg dizajnov} je verejný zoznam všetkých dostupných dizajnov.
Umožnuje vyhľadávanie pomovou textu alebo tagov.

\item
\textbf{Hlavný kanál} je chronologický zoznam aktivity v aplikácii
viditelný na hlavnej stránke aplikácie.
Aktivita predstavuje zdielanie nového dizajnu, vygenerovanie modela,
ale aj sociálnu interakciu (lajkovanie, komentovanie, sledovanie).

\end{itemize}

%===========================================================================
%===========================================================================
%===========================================================================
\section{Používateľské požiadavky}

Aplikácia bude rozlišovať následujúce typy používateľov:
\begin{itemize}
    \item \textbf{Guest} ---
        Implicitná rola pre neprihláseného používateľa.
        Môže prezerať katalóg dostupných dizajnov, 
        ale nemôže sťahovať vygenerované modely, 
        nahrávať nové dizajny, komentovať.
        Do aplikácie sa môže prihlásiť alebo zaregistrovať,
        čím získa väčšie právomoci.
    \item \textbf{User} ---
        Štandartná rola pre prihláseného používateľa.
        Umožňuje generovanie modelov zo zdielaných dizajnov 
        zadaním konkrétnych hodnôt parametrov a ich následné sťahovanie.
        Umožňuje sociálnu interakciu (lajkovanie, komentovanie, sledovanie) 
        s ostatními registrovanými používateľmi.
        Rola sa dá zvoliť pri registrácii.
    \item \textbf{Designer} ---
        Rola má všetky možnosti ako rola User.
        Umožňuje navyše zverejňovať dizajny.
        Dá sa zvoliť pri registrácii alebo rozšírením role User
        pri zmene údajov profilu.
    \item \textbf{Moderator} ---
        Rola správcu obsahu aplikácie.
        Umožnuje mazať komentáre, dizajny.
        Môže zablokovať používateľov s rolou User a Designer.
        Rola musí byť priradená administrátorom, nedá sa zvoliť pri registrácii.
    \item \textbf{Admin} ---
        Rola hlavného správcu aplikácie.
        Rola má všetky možnosti ako rola Moderator.
        Navyše môže zablokovať aj používateľov s rolou Moderator.
        Môže používateľom priradiť alebo odobrať rolu Moderator.
        Používateľ s rolou Admin existuje iba jeden.
\end{itemize}

\includegraphics[width=\linewidth]{use_case.png}


%===========================================================================
%===========================================================================
%===========================================================================
\section{Dátový model}

\includegraphics[width=\linewidth]{data_model.png}

\begin{itemize}
    \item \textbf{User} ---
        Obsahuje údaje o všetkých užívateľoch
\end{itemize}


%===========================================================================
%===========================================================================
%===========================================================================
\section{Technologické požiadavky}

\begin{itemize}

\item
Aplikácia typu SPA.

\item
Klient vytvorený v jazyku TypeScript
s využitím frameworku React s balíčkami React Router, React Bootstrap a Axios.


\item
Server vytvorený v jazyku TypeScript
v prostredí Node.js
s využitím balíčkov Express, Mongoose a JWT.
Server bude využívať lokálne bežiaci program OpenSCAD.

\item
Rozhlanie medzi klientom a serverom pomocou REST API.

\item
Databáza typu MongoDB.

\item
Hosting na AWS EC2.

\end{itemize}


%===========================================================================
%===========================================================================
%===========================================================================
\section{Časový plán}

\begin{itemize}
    \item Týždeň 5
    \begin{itemize}
        \item Inicializovanie projektu --- 7h
        \item Inicializovanie databáze --- 3h
        \item Registrácia a login --- 6h
        \item Renderovanie OpenSCAD na serveri --- 6h
    \end{itemize}
    \item Týždeň 6
    \begin{itemize}
        \item Pridávanie dizajnov --- 10h
        \item Generovanie modelov --- 8h
    \end{itemize}
    \item Týždeň 7
    \begin{itemize}
        \item Nahrávanie fotografii --- 6h
        \item Lajkovanie a komentovanie --- 5h
        \item Hlavný kanál --- 5h
        \item Katalóg dizajnov --- 4h
    \end{itemize}
    \item Týždeň 8
    \begin{itemize}
        \item Prezeranie 3D modelu v prehliadači --- 8h
        \item Funkcie moderátora a administrátora --- 4h
        \item Vyhľadávanie a filtrovanie --- 3h
    \end{itemize}
    \item Týždeň 9
    \begin{itemize}
        \item Hostovanie na AWS CP2 --- 10h
        \item Testovanie --- 2h
        \item Buffer pred odovzdaním beta --- 6h
    \end{itemize}
    \item Týždeň 10
    \begin{itemize}
        \item Emailové notifikácie --- 4h
        \item Cachovanie modelov --- 8h
    \end{itemize}
    \item Týždeň 11
    \begin{itemize}
        \item Testovanie --- 2h
        \item Buffer pred odovzdaním --- 6h
    \end{itemize}

\end{itemize}


\end{document}
